\title{Schlusswort}

{
\large
\setlength{\parskip}{\baselineskip}

\begin{center}
\textbf{Das letzte Wort}

- haben diesmal ausnahmsweise nicht die Lehrer, sondern wir, der Abiturjahrgang 2021 der Robert-Bosch-Schule Ulm
\end{center}

Wir blicken zurück auf drei gemeinsame Jahre, eine Menge an Klausuren und Arbeiten - und das alles während einer globalen Pandemie.

Jetzt ist es endlich so weit - wir können stolz die fertige Abizeitung präsentieren, an der wir gemeinsam als Abizeitungs-Team oftmals verzweifelt sind. Es ist ja fast ein Wunder, dass diese doch noch zustande gekommen ist, bei all den Hürden, die uns hierbei im Weg standen. Das lag vor allem an unserem eigenen Perfektionismus, der uns mehr als nur einmal im Weg stand. Trotz der anfangs geringen Teilnehmerzahl unserer Mitschüler bei den Rankings und den Steckbriefen, gelang es uns schlussendlich doch noch, einige zu motivieren und so eine relativ vollständige Zeitung zu erstellen. Es ist kein Geheimnis, dass einige von uns bestimmt mehr Zeit für die Abizeitung aufgebracht haben, als für das Abitur selbst zu lernen.

Nach diesen drei Jahren voller Höhen und Tiefen haben wir etliche Dankeschöns zu verteilen - an all diejenigen, die uns mit Rat und Tat zur Seite standen. Um uns alle zum und durchs Abi zu führen, haben viele verschiedene Menschen ihren Teil beigetragen - Familie, Freunde, Mitschüler - aber auch vor allem unsere Lehrerinnen und Lehrer, die bei jedem Wehwehchen trotz Pandemie immer ein offenes Ohr für uns hatten und ihr Bestes gegeben haben, den Fernunterricht für uns so angenehm wie möglich zu machen. Danke für interessante Unterrichtsgestaltung, Humor, alles Menschliche abseits des Lehrplans und auch für die Tage, an denen mal 10 Minuten früher Schluss war.

Oftmals kam jedem von uns die Situation fast aussichtslos vor, den ganzen Tag nur alleine zu Hause vor dem Laptop zu sitzen ist schließlich nicht besonders erfüllend. Nie konnte man wissen, wie die Inzidenz nächste Woche sein würde - würden wir nun Wechselunterricht (Tage- oder Wochenweise), komplett Fernunterricht oder doch sogar Präsenz haben? Bei diesem Regel-Wirrwarr war selbst der Informierteste unter uns irgendwann nur noch überfordert.  Gerade der Online-Unterricht stellte nicht nur uns, sondern auch unsere Lehrerinnen und Lehrer anfangs vor einige Probleme. Wir bekamen unsere Unterrichtsmaterialien über die unterschiedlichsten Wege - wie per E-Mail, Nextcloud, Onedrive, Zoom, Discord, BBB oder gar über eigens erstellte Web-Lösungen. Und als wir dann eine gemeinsame Plattform (Moodle) hatten, steckten wir auch schon fast in der zweiten Welle.

Trotz alldem haben wir es jetzt geschafft, erfolgreich unser Abitur zu bestehen und unsere Schulzeit neigt sich nun dem Ende zu.  Das Abizeitungs-Team hofft, dass sowohl die Schülerinnen und Schüler als auch die Lehrerinnen und Lehrer gerne an die gemeinsam verbrachte Zeit zurückdenken.

\begin{center}
\textbf{Die Redaktion der Abizeitung}
\end{center}
}

\null\hfill Leia Saumweber
